% !TEX TS-program = pdflatex
% !TEX encoding = UTF-8 Unicode

\documentclass{scrartcl} % use larger type; default would be 10pt

\usepackage[utf8]{inputenc} % set input encoding (not needed with XeLaTeX)

%%% PAGE DIMENSIONS
\usepackage{geometry} % to change the page dimensions
\geometry{a4paper} % or letterpaper (US) or a5paper or....

\usepackage{graphicx} % support the \includegraphics command and options

 %\usepackage[parfill]{parskip} % Activate to begin paragraphs with an empty line rather than an indent

%%% PACKAGES
\usepackage{booktabs} % for much better looking tables
\usepackage{array} % for better arrays (eg matrices) in maths
\usepackage{paralist} % very flexible & customisable lists (eg. enumerate/itemize, etc.)
\usepackage{verbatim} % adds environment for commenting out blocks of text & for better verbatim
\usepackage{subfig} % make it possible to include more than one captioned figure/table in a single float
\usepackage{color}
\usepackage{ulem}
\usepackage{amsthm}
\usepackage{amsmath}
\usepackage{amssymb}
\usepackage{graphicx}
\usepackage{hyperref}
\usepackage{tabularx}
\usepackage[numbers]{natbib} % for citeauthor


\title{Outline}
\author{Alex Baier \\ abaier@uni-koblenz.de}


\begin{document}
\maketitle

\section{Introduction}
Motivation. Related work. Solution. Evaluation.

\section{Foundations}

\subsection{Wikidata}

\subsection{Taxonomy}
\begin{itemize}
\item Ontology\\
	\citeauthor{Cimiano2009} \cite{Cimiano2009}
\item Taxonomy\\
	\citeauthor{Cimiano2009} \cite{Cimiano2009}
\item Connected taxonomy (maybe: consistent taxonomy)
\item Root class
\item Unlinked class
\item Problem statement
\end{itemize}

\subsection{Similarity}
\begin{itemize}
\item semantic similarity e.g. distributional similarity \\
	\citeauthor{Lin1998} \cite{Lin1998} \\
	\citeauthor{Rodriguez2003} \cite{Rodriguez2003}
\item geometrical similarity e.g. distance based-similarity, cosine similarity
\end{itemize}

\subsection{Similarity-based classification}
\citeauthor{Chen2009} \cite{Chen2009}\\
\citeauthor{Zhang2015} \cite{Zhang2015}

\subsection{Text processing}
\begin{itemize}
\item N-Gram \\
	\citeauthor{Jurafsky2014} \cite{Jurafsky2014}
\item Skip-Gram \\
	\citeauthor{Guthrie2006} \cite{Guthrie2006}
\item Counting-based word representations \\ 
	\citeauthor{Levy2015} \cite{Levy2015}
\item Predictive word representations \\
	\citeauthor{Levy2015} \cite{Levy2015}
\end{itemize}

\section{Ontology learning}
General concepts. Classification of considered problem in the task of ontology learning. Related work.\\
\citeauthor{Cimiano2009} \cite{Cimiano2009}\\
\citeauthor{Wong2012} \cite{Wong2012}\\
\citeauthor{dAmato16} \cite{dAmato16}\\
\citeauthor{Petrucci16} \cite{Petrucci16}

\section{Neural networks}
Notion of neural networks will be introduced.

\subsection{Recursive neural networks for graph representation}
\citeauthor{Scarselli2009} \cite{Scarselli2009}
\subsection{Deep neural networks for graph representation}
\citeauthor{Cao2016} \cite{Cao2016} \\
\citeauthor{Raghu2016} \cite{Raghu2016}
\subsection{Continuous Bag-of-Words}
\citeauthor{Mikolov2013} \cite{Mikolov2013}
\subsection{Skip-gram with negative sampling}
\citeauthor{Mikolov2013} \cite{Mikolov2013}\\
\citeauthor{Levy2015} \cite{Levy2015}\\
\citeauthor{Goldberg14} \cite{Goldberg14}

\subsection{Comparison}

\section{Algorithm}

\subsection{Baseline}
\begin{itemize}
\item Hyper parameters
\item Training data
\end{itemize}

\subsection{Supplementing with other resources}
e.g. Wikipedia

\section{Evaluation}

\subsection{Method}
\citeauthor{Dellschaft2006} \cite{Dellschaft2006}

\subsection{Generation of gold standard}

\subsection{Results}

\bibliographystyle{plainnat}
\bibliography{/home/alex/PycharmProjects/thesis/thesis/bibliography}


\end{document}
